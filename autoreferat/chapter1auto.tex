\chapter{Introduction}

\emph{Grid computing} is usually defined as sharing computational and data storage resources across organizational boundaries which can give a user much more computational or storage capacity. Grid computing in contrast to common distributed computing focus on large-scale resource sharing. The technology under grid computing provides access to a computational resources in a federated way, while preserving some rights of the owner. Requirements, standards and architecture, were proposed and published, e.g., by Foster et al. \cite{Foster2001, Foster2003} and such infrastructures are currently distinguished as "service" grids. It's non-trivial task to maintain scientific grid, thus specialists from the so-called national grid initiatives (NGI)\nomenclature{NGI}{National Grid Initiative} maintains and cooperates with similar grid initatives of other countries. In Europe these are coordinated, e.g., by European Grid Infrastructure (EGI)\nomenclature{EGI}{European Grid Infrastructure}. One of the largest project computed in these grid infrastructures are related to experiments of high-energy physics in order to process a large number of observed data in a reasonable time \cite{Bird2009}. The Worldwide Large Hadron Collider Computing Grid (WLCG)\nomenclature{WLCG}{Worldwide Large Hadron Collider Computing Grid} was designed to store and process almost 30 PetaBytes of data per year in the period of 2009-2013  \cite{Adamova2014}.

Another approach to grid computing is joining desktop computers from an individual user to form a voluntary or desktop grid. It was popularized by a project that tries to identify uncommon signals from space to search for extraterrestrial intelligence (SETI@Home)\footnote{\url{http://setiathome.ssl.berkeley.edu/}}\cite{Anderson2002}. And general-purpose frameworks were built in order to facilitate the development of projects that use a similar philosophy of computing on desktop computers, e.g., BOINC\nomenclature{BOINC}{Berkeley Open Infrastructure for Network Computing} \cite{Anderson2004} and others.

In recent years, the development of virtualization technologies has enhanced the availability of services that are provided by grid computing. It has additionally enabled an evolution of the so-called \emph{cloud computing}, in which computing resources can be rapidly provisioned and released with minimal management effort or service provider interaction. This implicates important feature of cloud-computing -- elasticity -- ability to scale up and down computing resources when required \cite{Mell2011}. The cloud computing is provided in several models, however, currently the scientific infrastructures offer mainly Infrastructure as a Service (IaaS)\nomenclature{IaaS}{Infrastructure as a Service}, which offers the whole virtual infrastructure including virtual machine and network accessible for user per request.

Applications that are computed within a grid or cloud infrastructure can be characterized by the quantity of tasks being performed, the size of the input data and the communication that needs to be carried out between concurrent tasks.
And using such characterization three main categories of the application model are recognized. 
\begin{itemize}
\item
The term High Throughput Computing (HTC)\nomenclature{HTC}{High Throughput Computing} is used for computation in which tasks take a long time. These are relatively loosely coupled and resources are used over a long period of time. Performance or capacity is usually mentioned  in operations or CPUs per month or year. Grid computing focus mainly on HTC.

%While HTC takes a long time, 
\item
The High Performance Computing (HPC)\nomenclature{HPC}{High Performance Computing} is usually characterized as a small number of tasks which need to communicate quite often. The tasks are relatively tightly coupled and can take shorter time than HTC. Performance is measured in operations per second (FLOPS)\nomenclature{FLOPS}{Floating-Point Operations per Second} \cite{Hager2010,Levesque2010}. The grid computing or cloud computing infrastructure can involve HPC servers or clusters.% and, thus, a job or task that requires such HPC hardware is scheduled and executed there.
\item
Many Task Computing (MTC)\nomenclature{MTC}{Many Task Computing} aims to bridge HTC and HPC. While the computation usually takes a shorter amount of time, the data exchange is in MB rather than in GB and it involves computing much more heterogeneous problems, which are not "happily" parallel \cite{Raicu2008}. %However, middleware for HPC or HTC, which are present in grid computing infrastructures, may introduce some shortcomings. 
%Therefore Raicu et al. proposed and implemented a prototype of task execution framework, which is suitable for MTC to prevent some shortcomings of HTC or HPC \cite{Raicu2008, Raicu2009,Raicu2010}.
\end{itemize}

With respect to technology development available in scientific infrastructures, this thesis focus not only on grid computing but also on cloud computing technology, which were available for scientific computing within grid infrastructures since 2012.
 %This work focuses on processing of medical information should give enhanced information, which can be analysed easily. Thus further analysis or synthesis of such information is beyond this work, so the aim is not to give particular results on some specific diseasies, pathologies etc., however, with cooperation of other experts it is a desired side effect.

%The author's work was published in a series of peer-reviewed papers of international journals and peer-reviewed conference proceedings \cite{kulhanek2009, kulhanek2010b, kulhanek2010c,  Kulhanek2014Parameters, Kulhanek2014Modeling, Kulhanek2014mefanet, Matejak2014sj} which are included in this work as appendices. The author's work and contribution was also presented in international conferences and published in the respective proceedings and transactions \cite{Kulhanek2010, Kulhanek2013c, kofranek2013hummod, Matejak2014}. The work was also popularized in local and regional conferences and their respective proceedings \cite{Kulhanek2008Mefanet, Sarek2009, kulhanek2009dd, Kulhanek2009Mefanet, Kulhanek2010d, Kulhanek2010Mefanet, Kulhanek2011, kulhanek2011dd, Kulhanek2012, Kulhanek2013b, Kulhanek2014, Kulhanek2012a}. The author contributed to the utility model, which was registered by the Czech Industrial Property Office \cite{Kofranek2014a}.

