\chapter{Introduction}

\emph{Grid computing} is usually defined as sharing computational and data storage resources across organizational boundaries which can give a user much more computational or storage capacity. Grid computing in contrast to common distributed computing focus on large-scale resource sharing. The technology under grid computing provides access to a computational resources in a federated way, while preserving some rights of the owner. Requirements, standards and architecture, were proposed and published, e.g., by Foster et al. \cite{Foster2001, Foster2003} and such infrastructures are currently distinguished as "service" grids. It's non-trivial task to maintain scientific grid, thus specialists from the so-called national grid initiatives (NGI)\nomenclature{NGI}{National Grid Initiative} maintains and cooperates with similar grid initatives of other countries. In Europe these are coordinated, e.g., by European Grid Infrastructure (EGI)\nomenclature{EGI}{European Grid Infrastructure}. One of the largest project computed in these grid infrastructures are related to experiments of high-energy physics in order to process a large number of observed data in a reasonable time \cite{Bird2009}. The Worldwide Large Hadron Collider Computing Grid (WLCG)\nomenclature{WLCG}{Worldwide Large Hadron Collider Computing Grid} was designed to store and process almost 30 PetaBytes of data per year in the period of 2009-2013  \cite{Adamova2014}.

Another approach to grid computing is joining desktop computers from an individual user to form a voluntary or desktop grid. It was popularized by a project that tries to identify uncommon signals from space to search for extraterrestrial intelligence (SETI@Home)\footnote{\url{http://setiathome.ssl.berkeley.edu/}}\cite{Anderson2002}. And general-purpose frameworks were built in order to facilitate the development of projects that use a similar philosophy of computing on desktop computers, e.g., BOINC\nomenclature{BOINC}{Berkeley Open Infrastructure for Network Computing} \cite{Anderson2004} and others.

In recent years, the development of virtualization technologies has enhanced the availability of services that are provided by grid computing. It has additionally enabled an evolution of the so-called \emph{cloud computing}, in which computing resources can be rapidly provisioned and released with minimal management effort or service provider interaction. This implicates important feature of cloud-computing -- elasticity -- ability to scale up and down computing resources when required \cite{Mell2011}. The cloud computing is provided in several models, however, currently the scientific infrastructures offer mainly Infrastructure as a Service (IaaS)\nomenclature{IaaS}{Infrastructure as a Service}, which offers the whole virtual infrastructure including virtual machine and network accessible for user per request.

With respect to technology development available in scientific infrastructures, this thesis focus not only on grid computing but also on cloud computing technology, which were available for scientific computing within grid infrastructures since 2012.
 %This work focuses on processing of medical information should give enhanced information, which can be analysed easily. Thus further analysis or synthesis of such information is beyond this work, so the aim is not to give particular results on some specific diseasies, pathologies etc., however, with cooperation of other experts it is a desired side effect.

%The author's work was published in a series of peer-reviewed papers of international journals and peer-reviewed conference proceedings \cite{kulhanek2009, kulhanek2010b, kulhanek2010c,  Kulhanek2014Parameters, Kulhanek2014Modeling, Kulhanek2014mefanet, Matejak2014sj} which are included in this work as appendices. The author's work and contribution was also presented in international conferences and published in the respective proceedings and transactions \cite{Kulhanek2010, Kulhanek2013c, kofranek2013hummod, Matejak2014}. The work was also popularized in local and regional conferences and their respective proceedings \cite{Kulhanek2008Mefanet, Sarek2009, kulhanek2009dd, Kulhanek2009Mefanet, Kulhanek2010d, Kulhanek2010Mefanet, Kulhanek2011, kulhanek2011dd, Kulhanek2012, Kulhanek2013b, Kulhanek2014, Kulhanek2012a}. The author contributed to the utility model, which was registered by the Czech Industrial Property Office \cite{Kofranek2014a}.

