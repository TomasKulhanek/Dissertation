
%%% Povinn� informa�n� strana diserta�n� pr�ce
\newpage
\begin{center}
\Large \textbf{Abstrakt (česky)}
\end{center} 
\normalsize
%\setlength\parindent{0mm}
%\setlength\parskip{2mm}
\selectlanguage{czech}
Práce se soustředí na vybrané oblasti biomedicínského výzkumu, které mohou profitovat ze současných výpočetních infrastruktur vybudovaných ve vědecké komunitě v evropském a světovém prostoru. Teorie výpočtu, paralelismu a distribuovaného počítání je stručně uvedena s ohledem na počítání v gridech a cloudech. Byla studována oblast výměny medicínských snímků a Gridový PACS systém byl propojen s existujícími distribuovanými systémy pro sdílení DICOM snímků. Další studovanou doménou byla věda týkající se lidského hlasu. Vzdálený přístup k aplikaci pro analýzu hlasu v reálném čase byl představen zároveň s úpravou protokolů pro vzdálenou plochu pro přenos zvukových nahrávek. To přináší možnost využití stávajících aplikací na dálku specialisty na hlas. 

Byl studován přístup tzv. systémové biologie v oblasti lidské fyziologie a patofyziologie. Bylo přispěno k metodologii modelování lidské fyziologie pro tvorbu komplexních modelů založených na akauzálním a objektově orientovaném modelovacím přístupu. Byly představeny metody pro studium parametrů pomocí technologie počítání v gridech a v cloudech. Proces identifikace parametrů středně komplexních modelů kardiovasculárního systému a komplexního modelu lidské fyziologie lze významně zrychlit při použití cloud computingu a dobrých výsledků lze dosáhnout v rozumném čase. Tato metoda umožňuje aplikovat parametrické studie ve fyziologickém a biologickém výzkumu. Toto může zlepšit praktické použití matematických modelů a identifikaci parametrů ve zdravotní péči.
%Technologie virtualizace zjedodušuje integraci doménově specifických aplikací do existujících výpočetních infrastruktur.


\textbf{Klíčová slova:}
gridové počítání, počítání v cloudu, výpočetní fyziologie, odhad parametrů, výměna medicínských snímků, analýza hlasového signálu

