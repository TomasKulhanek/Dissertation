
%%% Povinn� informa�n� strana diserta�n� pr�ce
\newpage
\begin{center}
\Large \textbf{Abstrakt (česky)}
\end{center} 
\normalsize
%\setlength\parindent{0mm}
%\setlength\parskip{2mm}
\selectlanguage{czech}
Práce se soustředí na vybrané oblasti biomedicínského výzkumu, které mohou profitovat ze současných výpočetních infrastruktur vybudovaných ve vědecké komunitě v evropském a světovém prostoru. Teorie výpočtu, paralelismu a distribuovaného počítání je stručně uvedena s ohledem na počítání v gridech a cloudech. Práce se zabývá oblastí výměny medicínských snímků a představuje propojení Gridového PACS systému s existujícími distribuovanými systémy pro sdílení DICOM snímků. Práce se dál zaměřuje na studium vědy týkající se lidského hlasu. Práce představuje vzdálený způsob přístupu k aplikaci pro analýzu hlasu v reálném čase pomocí úpravy protokolů pro vzdálenou plochu a pro přenos zvukových nahrávek. Tento dílčí výsledek ukazuje možnost využití stávajících aplikací na dálku specialisty na hlas. 

Oblast lidské fyziologie a patofyziologie byla studována pomocí přístupu tzv. systémové biologie. Práce přispívá v oblasti metodologie modelování lidské fyziologie pro tvorbu komplexních modelů založených na akauzálním a objektově orientovaném modelovacím přístupu. Metody pro studium parametrů byly představeny pomocí technologie počítání v gridech a v cloudech. Práce ukazuje, že proces identifikaci parametrů středně komplexních modelů kardiovasculárního systému a komplexního modelu lidské fyziologie lze významně zrychlit při použití cloud computingu a dobrých výsledků lze dosáhnout v rozumném čase. Tato metoda umožňuje aplikovat parametrické studie ve fyziologickém a biologickém výzkumu. Toto může zlepšit praktické použití matematických modelů a identifikaci parametrů ve zdravotní péči do budoucna.
%Technologie virtualizace zjedodušuje integraci doménově specifických aplikací do existujících výpočetních infrastruktur.


\textbf{Klíčová slova:}
gridové počítání, počítání v cloudu, výpočetní fyziologie, odhad parametrů, výměna medicínských snímků, analýza hlasového signálu

