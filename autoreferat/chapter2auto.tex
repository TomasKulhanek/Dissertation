\chapter{Hypothesis}

The hypothesis of this thesis is that the technologies that relate to grid computing and cloud computing may improve the processing of medical information in order to perform demanding tasks that are almost impossible or require onerous effort to achieve, using classical local or institutional resources.

The particular goals of this thesis are:
\begin{itemize}
\item To study the latest achievements in the field of exchanging medical images and  possible improvements using the grid computing and cloud computing technology.
%\item Analysis and processing of voice signal to support clinical and educational applications.
%\item analysis and mathematical modeling and simulation of biological systems
\item To identify use cases in other fields of biomedicine which are suitable to utilizing the power of grid computing and cloud computing infrastructure.
\item To develop and test the prototype application that utilizes grid or cloud technologies.
\end{itemize}

This thesis tries to discuss the hypothesis in different areas of biomedical research and its application which were identified during the work. (1) the exchange and processing of medical images, (2) the analysis of human voice and (3) the modeling and simulation of human physiology.

It tries to find answers to the following additional questions:
\begin{itemize}
\item \emph{Is it beneficial to utilize grid computing and cloud computing technology for the processing of medical information and how do we do this?} %The goal was to identify use cases in several fields of biomedicine which are suitable to utilize the power of grid-computing and cloud-computing infrastructure. Develop and test prototype application and compare achievements.%In the field of medicine, such large ammount of data were exchanged using DICOM format and protocol explained in section \ref{sec:imaging} and the issue was to integrate it with computing infrastructure and tools designed for the purpose of science in particle physics.
%\item \emph{What are the limitation of utilizing these technologies?} In the time of starting the work on this thesis, the majority of services and tools in grid-computing environment were developed for the purpose of computation of particle physics experiments, however, infrastructure were designed to be open for any scientific domain.
\item \emph{What are the limitations of processing medical information in grid or cloud?} 
\item \emph{How can the grid computing and cloud computing influence the direction of biomedical research?} There was an idea that grid computing technology inspires the current architecture of distributed systems,  e.g., exchanging medical images, and influences the direction of information systems in hospitals. 
\end{itemize}

