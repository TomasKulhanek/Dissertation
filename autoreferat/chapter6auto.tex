\chapter{Conclusion}
\label{sec:conclusion}

This thesis presents the infrastructure, which, thanks to virtualization technology, joined several domain-specific tools in the field of sharing and processing medical images, performing real-time voice analysis and simulating human physiology.  

A seamless integration of grid-based PACS system was established with the current distributed system in order to share DICOM medical images. Access to real-time voice analysis application via remote desktop technology brings this type of service to any computer that can connect to the Internet. A system and portal to support the analysis and building of complex models of human physiology in the phase of parameter estimation and parameter sweep was introduced. Furthermore, additional computational nodes can be flexibly joined by starting the prepared virtual machines in cloud computing deployment. 

The methodology of building complex models of human physiology was contributed with the use of acausal and object-oriented modeling techniques. Methods for conducting a parameter study were shown, as well as the parameter study of complex models that gain substantial speedup by utilizing cloud computing deployment, which makes such kinds of complex studies applicable in physiological and biological research.

