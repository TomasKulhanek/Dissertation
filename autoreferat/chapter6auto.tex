\chapter{Conclusion}
\label{sec:conclusion}

This thesis presents the infrastructure, which, thanks to virtualization technology, joined several domain-specific tools in the field of sharing and processing medical images, performing real-time voice analysis and simulating human physiology.  

A seamless integration of grid-based PACS system was established with the current distributed system in order to share DICOM medical images. The grid-based solution brings robustness against the mentioned problems. 

Access to real-time voice analysis application via remote desktop technology brings this type of service to any computer that can connect to the Internet. This connects  voice therapists and voice pedagogues in different areas of the Czech Republic and Slovakia to analyze the voice in non-invasive  way and to see e.g. the progress of the voice training methods.

A system and portal was introduced in order to support the analysis and building of complex models of human physiology in the phase of parameter estimation and parameter sweep. Furthermore, additional computational nodes can be joined flexibly by starting the prepared virtual machines in cloud computing deployment. 

The methodology of building complex models of human physiology was contributed with the recommendation and implementation of acausal and object-oriented modeling techniques. Methods for conducting a parameter study were shown, as well as the parameter study of complex models that gain substantial speedup by utilizing cloud computing deployment. This makes such kinds of complex studies applicable in physiological and biological research in future.

