\chapter{Introduction}

\emph{Grid computing} is usually defined as sharing computational and data storage resources across organizational boundaries. In recent years, the development of virtualization technologies has enhanced the availability of services that are provided by grid computing. It has additionally enabled an evolution of the so-called \emph{cloud computing}, which also utilizes a virtual environment on powerful computing infrastructures. Based on the development of technologies and the philosophy of providing them to end users, this thesis focuses on the multidisciplinary research related to grid computing, as well as to cloud computing. It discusses its utilization in biomedical research and its application in relation to the processing of medical information.

The term "medical information" is too broad and further work in this thesis focuses on the following selected areas:
(1) the exchange and processing of medical images, (2) the analysis of human voice and (3) the modeling and simulation of human physiology. %This work focuses on processing of medical information should give enhanced information, which can be analysed easily. Thus further analysis or synthesis of such information is beyond this work, so the aim is not to give particular results on some specific diseasies, pathologies etc., however, with cooperation of other experts it is a desired side effect.

The author's work was published in a series of peer-reviewed papers of international journals and peer-reviewed conference proceedings \cite{kulhanek2009, kulhanek2010b, kulhanek2010c,  
Kulhanek2014Parameters, Kulhanek2014Modeling, Kulhanek2014mefanet, Matejak2014sj} which are included in this work as appendices.
The author's work and contribution was also presented in international conferences and published in the respective proceedings and transactions
\cite{Kulhanek2010, Kulhanek2013c, kofranek2013hummod, Matejak2014}. The work was also popularized in local and regional conferences and their respective proceedings \cite{Kulhanek2008Mefanet, Sarek2009, kulhanek2009dd, Kulhanek2009Mefanet, Kulhanek2010d, Kulhanek2010Mefanet, Kulhanek2011, kulhanek2011dd, Kulhanek2012, Kulhanek2013b, Kulhanek2014, Kulhanek2012a}. The author contributed to the utility model, which was registered by the Czech Industrial Property Office \cite{Kofranek2014a}.

\section{Thesis Goal}
\label{sec:goal}
The hypothesis of this thesis is that the technologies that relate to grid computing and cloud computing may improve the processing of medical information in order to perform demanding tasks that are almost impossible or require onerous effort to achieve, using classical local or institutional resources.
The particular goals of this thesis were:
\begin{itemize}
\item To study the latest achievements in the field of exchanging medical images and  possible improvements using the grid computing and cloud computing technology.
%\item Analysis and processing of voice signal to support clinical and educational applications.
%\item analysis and mathematical modeling and simulation of biological systems
\item To identify use cases in other fields of biomedicine which are suitable to utilizing the power of grid computing and cloud computing infrastructure.
\item To develop and test the prototype application that utilizes grid or cloud technologies.
\end{itemize}
 
This thesis tries to discuss the hypothesis in different areas of biomedical research and its application. It tries to find answers to the following additional questions:
\begin{itemize}
\item \emph{Is it beneficial to utilize grid computing and cloud computing technology for the processing of medical information and how do we do this?} When work on this thesis begun, grid computing was believed to be an answer to scalability issues, e.g., for exchanging large amounts of data or carrying out demanding long-term computation. %The goal was to identify use cases in several fields of biomedicine which are suitable to utilize the power of grid-computing and cloud-computing infrastructure. Develop and test prototype application and compare achievements.%In the field of medicine, such large ammount of data were exchanged using DICOM format and protocol explained in section \ref{sec:imaging} and the issue was to integrate it with computing infrastructure and tools designed for the purpose of science in particle physics.
%\item \emph{What are the limitation of utilizing these technologies?} In the time of starting the work on this thesis, the majority of services and tools in grid-computing environment were developed for the purpose of computation of particle physics experiments, however, infrastructure were designed to be open for any scientific domain.
\item \emph{What are the limitations of processing medical information in grid or cloud?} 
\item \emph{How can the grid computing and cloud computing influence the direction of biomedical research?} There was an idea that grid computing technology inspires the current architecture of distributed systems,  e.g., exchanging medical images (explained in section \ref{sec:imaging}) and influences the direction of information systems in hospitals. 
\end{itemize}

Answers to these questions %, which are based on the following chapters,
are summarized in section \ref{sec:discussion}.

\section{Thesis Contribution}
The author claims that the following contribution was made to state-of-the-art biomedical informatics and computational biology.
\begin{itemize}
\item Proposal of a grid infrastructure and pilot implementation of a grid-based system that exchanges medical images that are integrated with an existing distributed systems. The results were published by Kulhánek and Šárek in \cite{kulhanek2009} and popularized in \cite{Kulhanek2008Mefanet,Sarek2009,kulhanek2009dd}. The author of this thesis customized the existing project, Globus MEDICUS, and deployed it in servers that are networked via the academic network, CESNET, and integrated with the existing regional PACS\nomenclature{PACS}{Picture Archiving and Communication System} system. Other co-author coordinated the work with selected hospitals and operators of the PACS system.
\item Pilot implementation of a more generic infrastructure as a service for the community within the biomedical research published by Kulhánek in \cite{kulhanek2010c, kulhanek2011dd}. The author of this thesis proposed the idea of consolidating and sharing physical resources in order to provide a virtual environment for the specific needs of particular use-cases. The pilot infrastructure was tested on examples of selected research projects.
\item Proposal of a system and implementation of a remote service for the real-time analysis of a human voice. The results were published by Kulhánek et al. in \cite{kulhanek2010b} and popularized in \cite{Kulhanek2010d, Kulhanek2012}. The author of this thesis designed and customized the existing network protocol in order to transfer voice signal losslessly. The author also deployed the application on a remote virtual server. Other co-authors implemented the algorithms and application in order to analyze voice signal.
\item Improved methodology for the modeling of complex physiological systems published by Kulhánek, Mateják, Kofránek et al. in \cite{Kulhanek2014Modeling, Kulhanek2014mefanet, kofranek2013hummod, Matejak2014}. The author of this thesis contributed to the idea of building complex mathematical models from the basic components and keeping them in an understandable and maintainable form. Additionally, the author implemented several basic blocks and models of a pulsatile cardiovascular system in Modelica language. The other co-authors implemented the library in order to model physiology, using an integrative approach. They also implemented the complex models, which integrated different domains together.
\item Design and implementation of a system to estimate the parameters of complex mathematical models in order to validate and calibrate models of the human physiology were published by Kulhánek et al. in \cite{Kulhanek2014Parameters} and gradual development of related technologies were published and popularized in \cite{Kulhanek2010, Kulhanek2013c, Kulhanek2011, Kulhanek2014}. The author of this thesis designed architecture for a distributed parameter estimation, integrated models and implemented a pilot deployment, which  utilized a scientific cloud computing infrastructure. Other co-authors implemented complex models of the human physiology in Modelica language and tested several algorithms for parameter estimation.
\item Improved mathematical model of oxygen, carbon dioxide and hydrogen ion binding to hemoglobin published by Mateják et al. in \cite{Matejak2014sj}. The author of this thesis implemented this model in Modelica language and identified its parameters. Other co-authors analyzed and proposed a new mathematical model, based on the basic physical and chemical laws and in relation to previously published studies.
\item Simulation of complex models of the human physiology as part of a virtual simulator on portable and mobile devices, utilizing cloud computing published by Kulhánek et al. in \cite{Kulhanek2013c,Kulhanek2013b}. The author of this thesis contributed to the idea of a hybrid architecture of web simulators. This utilizes the infrastructure for parameter estimation in order to simulate complex models remotely and process/visualize the results locally. Other co-authors implemented complex models of the human physiology and simulation scenarios for educational purposes.
\item Virtual patient simulator prototype was registered as a utility model by the Industrial Property Office in the Czech Republic \cite{Kofranek2014a}. The author of this thesis designed and developed a specific module to control the multiple instances of a virtual simulator within a virtual classroom via a web server application. Other co-authors designed and implemented models of the human physiology and clinically relevant educational scenarios. They also implemented a 3D visualization of selected scenarios using game engine Unity 3D\footnote{\url{http://unity3d.com/} accessed March 2015}.
\end{itemize}

\section{Thesis Structure}
This thesis is interdisciplinary and, therefore, the following chapters cover the topics from a technical and computer-science point of view. They also touch on some topics that are related to medical science.
Chapter \ref{sec:stateoftheart} provides an overview of the state-of-the-art theory of computation, parallel computation and distributed computing especially grid computing and cloud computing. 

%The chapter \ref{sec:methods} describes general methods available for integrating different technologies and specific methods used to obtain further results. 

Introduction to selected areas of biomedical research domains and related particular methods are in chapter \ref{sec:methods}. Sharing medical images is introduced in section \ref{sec:imaging}, voice science is introduced in section \ref{sec:voice} and computational biology is introduced in section \ref{sec:models}.

Chapter \ref{sec:results} summarizes the general results that were obtained by the research methods in specific areas of biomedical research and applications. Chapter \ref{sec:conclusion} discusses the achievements and answers of the hypothesis, as well as the questions thate were stated at the beginning of the work. It also recommends further areas for research.

The appendices contain selected papers \cite{kulhanek2009,kulhanek2010b,kulhanek2010c,Kulhanek2014Parameters, Kulhanek2014Modeling, Kulhanek2014mefanet, Matejak2014sj} that are most relevant to the topic of this thesis. These were published in international peer-reviewed journals or in peer-reviewed conference proceedings:

\textbf{Appendix~\ref{app:processing}} is the paper \cite{kulhanek2009} \emph{Processing of Medical Images in Virtual Distributed Environment}, published by ACM as part of the proceedings of the 2009 Euro American Conference on Telematics and Information Systems: New Opportunities to Increase Digital Citizenship.

\textbf{Appendix~\ref{app:remote}} is the paper \cite{kulhanek2010b} \emph{Remote Analysis of Human Voice – Lossless Sound Recording Redirection}, published in Analysis of Biomedical Signals and Images, Proceedings of 20th International EURASIP Conference (BIOSIGNAL).

%\textbf{Appendix~\ref{app:fromeducational}} is the paper \cite{Kulhanek2011} \emph{From Educational Models Towards Identification of Physiological Systems} published by Institute of Biostatistics and Analyses of Masary University in Brno, Czech Republic in the proceedings Mefanet Report 04, Efficient multimedia teaching tools in medical education.  

\textbf{Appendix~\ref{app:infrastructure}} is the paper \cite{kulhanek2010c} \emph{Infrastructure for Data Storage and Computation in Biomedical Research}, published by Euromise s.r.o. in the European Journal of Biomedical Informatics.

\textbf{Appendix~\ref{app:parameter}} is the paper \cite{Kulhanek2014Parameters} \emph{Parameter Estimation of Complex Mathematical Models of Human Physiology Using Remote Simulation Distributed in Scientific Cloud}, published in the IEEE Xplore Digital Library as part of the proceedings of the 2014 IEEE-EMBS International Conference on Biomedical and Health Informatics.

\textbf{Appendix~\ref{app:modeling}} is the paper \cite{Kulhanek2014Modeling} \emph{Modeling of Short-term Mechanism of Arterial Pressure Control in the Cardiovascular System: Object-oriented and Acausal Approach}, published by ELSEVIER in Computers in Biology and Medicine 2014, \textbf{IF(2013): 1.475}.

\textbf{Appendix~\ref{app:simplemodelsd}} is the paper \cite{Kulhanek2014mefanet} \emph{Simple Models of the Cardiovascular System for Educational and Research Purposes} published in Mefanet Journal 2014.

\textbf{Appendix~\ref{app:adair}} is the paper \cite{Matejak2014sj} \emph{Adair-based Hemoglobin Equilibrium with Oxygen, Carbon Dioxide and Hydrogen Ion Activity}, published in Scandinavian Journal of Clinical and Laboratory Investigation 2014, \textbf{IF(2013): 2.009}.

