\chapter{Voice Science}
\label{sec:voice}
With introduction of objective data analysis and laryngoscopy methods the voice science emphasized the cooperation among  laryngologist, speech pathologist and voice teacher.
The human voice ranges from 50 Hz to something about 1000 Hz, but there are large  individual variation. For analysis of digitally recorded voice, either habitual or singing, the Discrete Fourier Transformation(DFT) is used to produce frequency and amplitude analysis of recorded input voice samples. One of the most used class algorithm to compute DFT is class of Fast Fourier Transformation with computational complexity $O(n \log(n))$ \cite{Cooley1965,Frigo2005} and parallel version of the algorithms may introduce additional speedup for larger samples of analyzed data \cite{Gupta1993,Takahashi2003}. The result of analysis can be visualized in a voice range profile and there can be seen significant difference between untrained and trained voice as well as quantitatively seen some disorders  \cite{DeLeoLeBorgne2002,wuyts2003effects}.

Other methods to analyse vocal chords is laryngoscopy. The videostroboscopy and high speed video in laryngoscope methods produce video for analysing the real movement of vocal chords. The videokymography method introduced by Švec et al. complements the videostroboscopy and allows to visualize and analyze movement of vocal cords recorded by high speed camera on standard TV or monitor with an artificial image built from recorded sequence of selected section \cite{Svec1996,Svec2007}. 

In case of recorded sound and further analysis there is a question about how such a service can be integrated in grid-computing or cloud-computing environment to provide access to a complex application for non-technical voice specialists. Additionally, the analytical software was already developed and calibrated for selected sorts of microphones in MS Windows platform \cite{Fric2007,Fric2012}. Therefore I proposed and implemented a method that provides access to the analytical software remotely. The section \ref{sec:methodsvoice} describes how the analytical software was customized with a remote desktop protocol (RDP). Results are described in section \ref{sec:resultsvoice}. Similar approach might be used for processing the video recordings from laryngoscope, however, the practical limits are discussed in section \ref{sec:conclusion}. 


\section{Methods for remote analysis of human voice}
\label{sec:methodsvoice}
%One method to provide access to specialized service is via remote access protocols. Secure Shell (SSH) is used to establish secure channel via unsecured network (e.g. the Internet) from SSH client to SSH server providing e.g. remote command-line, remote command execution etc. It is one of the basic method to access the grid infrastructure and submit computational jobs. 
%Another method is to have a web portal and this web portal based on user's input executes command-line batch scripts over SSH, or it can utilize web services to submit computational job.
Terminal access to some remote computational capabilities, e.g. remote command-line or remote execution is another integration strategy to some remote infrastructure. Secure Shell (SSH) is used to establish secure channel via unsecured network (e.g. the Internet) from SSH client to SSH server and it is basic method to access grid-computing infrastructure. 

Remote Desktop Protocol(RDP) is a proprietary protocol for desktop sharing developed primarly in Microsoft Windows platform, however, today clients and servers exists for several other platforms. Next to remote command-line, remote execution it allows to access remote graphical desktop environment. %VNC is sharing protocol to access remote graphical desktop environment\cite{Richardson1998}. 
The software for parameterized Voice Range Profile (ParVRP) and Voice Range Profile in Real time (RealVoiceLab) was already developed and calibrated for selected sorts of microphones in MS Windows platform by Fric et al.\cite{Fric2007,Fric2012}. The implementation is done in MATLAB environment utilizing Signal Processing Toolbox\footnote{\url{http://www.mathworks.com/products/signal/} accessed February 2015}and compiled with MATLAB Compiler and distributed as an executable.

Instead of migrating the application into some compatible platform for grid-middleware, a virtual machine was introduced and access to the software is provided via RDP protocol. RDP itself contains redirection of several services, e.g. sound recording or drive access. Because the default sound recording redirection introduces some sound degradation without control, I proposed, implemented and integrated the custom RDP plugin with the ParVRP and RealVoiceLab software to redirect the sound recording without loss of information. 

The computation of frequencies and amplitude from the recorded samples utilizes Fast Fourier Transformation which has time complexity $O(n\log(n))$. Thus the computational complexity and theoretical speedup is not primary reason for making such application distributed. 

This type of application can be packaged as virtual machine template and configured within different types of cloud infrastructures and together with a script or web portal the on-demand deployment can be automated.

\section{Results}
\label{sec:resultsvoice}

The paper \cite{kulhanek2010b} \emph{Remote Analysis of Human Voice--Lossless Sound Recording Redirection} in the Appendix~\ref{app:bio} published technical details and results of such implementation.

Additionally the custom RDP plugin with the ParVRP and RealVoiceLab software to redirect the sound recording without loss of information was packaged as a virtual machine template and  deployed in the pilot virtual infrastructure next to the test instance of Globus MEDICUS.

The virtual machine template was also deployed to different cloud computing infrastructures. One to the Amazon EC2\footnote{\url{http://aws.amazon.com/ec2/} accessed February 2015} and second to the pilot scientific cloud launched in the begining of 2012 --MetaCloud\footnote{\url{http://www.metacentrum.cz/en/cloud/} accessed February 2015}.

 A comparison of these two was presented to the users and staff of CESNET and EGI in EGI Technical Forum 2012\cite{Kulhanek2012a} with focus on providing such science as a service.
 