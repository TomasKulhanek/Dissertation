
\nobreak\vbox to 0.49\vsize{

\begin{center}
\Large \textbf{Abstract}
\end{center} 

\normalsize
\setlength\parindent{0mm}
\setlength\parskip{2mm}
\selectlanguage{english}
\textbf{Title:}
Utilization of GRID technology in processing medical information

\textbf{Author:}
Mgr. Tomáš Kulhánek

\textbf{Department:}
Institute of pathological Physiology

\textbf{Supervisor:}
Ing. Milan Šárek, CSc., CESNET z.s.p.o

\textbf{Consultant:}
Doc. MUDr. Jiří Kofránek, CSc.

This thesis focuses on selected areas of biomedical research in order to benefit from current computational infrastructures established in scientific community in european and global area. The theory of computation, parallelism and distributed computing, with focus on grid computing and cloud computing, is briefly introduced. Exchange of medical images was studied and a seamless integration of grid-based PACS system was established with the current distributed system in order to share DICOM medical images. Voice science was studied and access to real-time voice analysis application via remote desktop technology was introduced using customized protocol to transfer sound recording. This brings a possibility to access current legacy application remotely by voice specialists. 
%A system and portal was introduced in order to estimate the parameters of the model and perform parameter study. Additionally, 

The systems biology approach within domain of human physiology and pathophysiology was studied. Modeling methodology of human physiology was improved in order to build complex models based on acausal and object-oriented modeling techniques. Methods for conducting a parameter study (especially parameter estimation and parameter sweep) were introduced using grid computing and cloud computing technology. The identification of parameters gain substantial speedup by utilizing cloud computing deployment when performed on medium complex models of cardiovascular system and complex models of human physiology. This makes such kind of study applicable in order to perform identification of physiological system in reasonable time for physiological and biological research and good results are available in a reasonable time. This can improve practical usage of mathematical models in healthcare.
%support parameter estimation and parameter sweep. 

% abstrakt v rozsahu 80-200 slov v angli�tin�; nejedn� se v�ak o p�eklad
% zad�n� diserta�n� pr�ce

\textbf{Keywords:}
grid computing, cloud computing, computational physiology, systems biology, parameter estimation, medical image exchange, voice signal analysis
% 3 a� 5 kl��ov�ch slov v angli�tin�

\vss}
