
%%% Povinn� informa�n� strana diserta�n� pr�ce

\vbox to 0.5\vsize{
\setlength\parindent{0mm}
\setlength\parskip{5mm}
\selectlanguage{czech}
\textbf{Název práce:}
Využití technologie GRID při zpracování medicínské informace

\textbf{Autor:}
Mgr. Tomáš Kulhánek

\textbf{Katedra,ústav: }
Ústav patologické fyziologie 1.LFUK  

\textbf{Školitel:}
Ing. Milan Šárek, CSc., CESNET z.s.p.o.

\textbf{Konzultant:}
Doc. MUDr. Jiří Kofránek, CSc.

\textbf{Abstrakt:}

Práce prezentuje výzkum a výsledky využití technologií, které umožňují sdílet výpočetní a úložné kapacity a to v oblasti biomedicínského výzkumu. Vedle technologie GRID se rozvíjí virtualizačních technologie (VMWare, XEN, ...), které dodali distribuovaným systémům novou vlastnost zdání vlastnictví a přímé kontroly konfigurovatelné infrastruktury jako služby, jež se dnes shrnují pod společný pojem CLOUD computing. V práci jsou diskutovány teoretické limity distribuovaných systémů a paralelních výpočtů v nich tak i praktické výsledky ve vybraných oblastech. V oblasti výměny medicínských snímků a souvisejících zdravotních záznamů byla ukázána snadná integrovatelnost se stávajícími systémy při respektování požadavků na bezpečnost dat. V oblasti analýzy a zpracování lidského hlasu v reálném čase byla ukázána možnost poskytování nadstandardních výpočetních služeb pro komunitu uživatelů. V oblasti modelování fyziologických systémů je prezentován systém pro odhad parametrů a identifikaci fyziologických systémů komplexních modelů, které by byli obtížně řešitelné za použití klasických metod.

\textbf{Klíčová slova:}

grid-computing, cloud-computing, matematické modelování, výpočetní fyziologie, fonetogram
% 3 a� 5 kl��ov�ch slov

\vss}
