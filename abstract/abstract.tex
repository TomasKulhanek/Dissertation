
%%% Povinn� informa�n� strana diserta�n� pr�ce

\vbox to 0.5\vsize{
\begin{center}
\Large \textbf{Abstrakt (česky)}
\end{center} 
\normalsize
\setlength\parindent{0mm}
\setlength\parskip{2mm}
\selectlanguage{czech}
\textbf{Název práce:}
Využití technologie GRID při zpracování medicínské informace

\textbf{Autor:}
Mgr. Tomáš Kulhánek

\textbf{Katedra,ústav: }
Ústav patologické fyziologie 1.LFUK  

\textbf{Školitel:}
Ing. Milan Šárek, CSc., CESNET z.s.p.o.

\textbf{Konzultant:}
Doc. MUDr. Jiří Kofránek, CSc.

Práce se soustředí na vybrané oblasti biomedicínského výzkumu, které mohou profitovat ze současných výpočetních infrastruktur vybudovaných ve vědecké komunitě v evropském a světovém prostoru. Teorie výpočtu, paralelismu a distribuovaného počítání je stručně uvedena s ohledem na počítání v gridech a cloudech. Byla studována oblast výměny medicínských snímků a Gridový PACS systém byl propojen s existujícími distribuovanými systémy pro sdílení DICOM snímků. Další studovanou doménou byla věda týkající se lidského hlasu. Vzdálený přístup k aplikaci pro analýzu hlasu v reálném čase byl představen zároveň s úpravou protokolů pro vzdálenou plochu pro přenos zvukových nahrávek. To přináší možnost využití stávajících aplikací na dálku specialisty na hlas. 

Byl studován přístup tzv. systémové biologie v oblasti lidské fyziologie a patofyziologie. Bylo přispěno k metodologii modelování lidské fyziologie pro tvorbu komplexních modelů založených na akauzálním a objektově orientovaném modelovacím přístupu. Byly představeny metody pro studium parametrů pomocí technologie počítání v gridech a v cloudech. Proces identifikace parametrů středně komplexních modelů kardiovasculárního systému a komplexního modelu lidské fyziologie lze významně zrychlit při použití cloud computingu a dobrých výsledků lze dosáhnout v rozumném čase. Tato metoda umožňuje aplikovat parametrické studie ve fyziologickém a biologickém výzkumu. Toto může zlepšit praktické použití matematických modelů a identifikaci parametrů ve zdravotní péči.

%This thesis focuses on selected areas of biomedical research in order to benefit from current computational infrastructures established in scientific community in european and global area. The theory of computation, parallelism and distributed computing, with focus on grid computing and cloud computing, is briefly introduced. A seamless integration of grid-based PACS system was established with the current distributed system in order to share DICOM medical images. Access to real-time voice analysis application via remote desktop technology brings this type of service to any computer that can connect to the Internet. 
%%A system and portal was introduced in order to estimate the parameters of the model and perform parameter study. Additionally, 
%The modeling methodology was contributed in order to build complex models based on  acausal and object-oriented modeling techniques. Methods for conducting a parameter study were shown, especially parameter estimation and parameter sweep. Parameter study of complex models gain substantial speedup by utilizing cloud computing deployment, which makes such kinds of complex studies applicable in physiological and biological research and have potential to improve such usage in healthcare.
%%support parameter estimation and parameter sweep. 
%
%The multidisciplinary projects were solved within the cooperation between association CESNET, First Faculty of Medicine of Charles University in Prague and Academy of Performing Arts in Prague. The work was partially supported by the project of Ministry of Education, Youth and Sport of the Czech Republic MSM6383917201, by the projects of Development Fund of CESNET 2009/361, 2010/384, 2011/423 and 2011/431 and by the project of Ministry of Industry and Business of the Czech Republic MPO FR-TI3/869.




%Práce prezentuje výzkum a výsledky využití technologií, které umožňují sdílet výpočetní a úložné kapacity a to v oblasti biomedicínského výzkumu. Vedle technologie GRID se rozvíjí virtualizačních technologie (VMWare, XEN, ...), které dodali distribuovaným systémům novou vlastnost zdání vlastnictví a přímé kontroly konfigurovatelné infrastruktury jako služby, jež se dnes shrnují pod společný pojem CLOUD computing. V práci jsou diskutovány teoretické limity distribuovaných systémů a paralelních výpočtů v nich tak i praktické výsledky ve vybraných oblastech. V oblasti výměny medicínských snímků a souvisejících zdravotních záznamů byla ukázána snadná integrovatelnost se stávajícími systémy při respektování požadavků na bezpečnost dat. V oblasti analýzy a zpracování lidského hlasu v reálném čase byla ukázána možnost poskytování nadstandardních výpočetních služeb pro komunitu uživatelů. V oblasti modelování fyziologických systémů je prezentován systém pro odhad parametrů a identifikaci fyziologických systémů komplexních modelů, které by byli obtížně řešitelné za použití klasických metod.

\textbf{Klíčová slova:}
gridové počítání, počítání v cloudu, výpočetní fyziologie, systémová biologie, odhad parametrů, výměna medicínských snímků, analýza hlasového signálu
\vss}
